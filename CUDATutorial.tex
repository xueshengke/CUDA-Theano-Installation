\documentclass[UTF8,a4paper,12pt]{article}
\usepackage[left=2.50cm, right=2.50cm, top=2.50cm, bottom=2.50cm]{geometry}

%\usepackage{times}
%\usepackage{mathpazo}
%\usepackage{fourier}
\usepackage{charter}
%\usepackage{helvet}

\usepackage{amsmath, amsfonts, amssymb}	% math equations, symbols
\usepackage[english]{babel}
\usepackage{color}		% color content
\usepackage{graphicx}	% import figures
\usepackage{url}		% hyperlinks
\usepackage{bm}			% bold type for equations
\usepackage{hyperref}	% bookmarks
\hypersetup{bookmarks, unicode}	% unicode
\usepackage{listings}
\usepackage{xcolor}
\lstset{frame = shadowbox, basicstyle = \small,
%	numbers = left, 
%	numberstyle = \small, 
	stepnumber = 1,
	breaklines = true, tabsize = 4
}

\newcommand{\code}[1]{{\color{blue} \textsf{{\$} {#1}} } }

\title{Tutorial for CUDA 7.5 Installation \\ Based on CentOS 7}
\author{ Xue Shengke \thanks{Contact e-mail: xueshengke@zju.edu.cn} }
\date{\today}

\begin{document}
\maketitle

\section{Prepare CentOS 7 distribution, with Python 2.7.* already installed}
Internet connection available is also required.

\section{Install NVIDIA graphic driver for Linux system}
NVIDIA graphic driver for Linux system will be installed individually before the CUDA installation, so we download the driver according to its actual type and proper operating system (Linux, 64bit / 32bit). \\
Taking GT 640 as an instance, we download NVIDIA-Linux-x86\_64-361.42.run.~Open a terminal (better as root),
\begin{lstlisting}
# uname -r
3.10.0-327.el7.x86_64 ; the distribution varies from person to person, so remember it 
# yum install gcc kernel-devel kernel-headers
\end{lstlisting}
Add two lines in both files as follows,
\begin{lstlisting}
# vim /etc/modprobe.d/blacklist.conf
# vim /lib/modprobe.d/dist-blacklist.conf
...
blacklist nouveau
options nouveau modeset=0
\end{lstlisting}
Remake initramfs image,
\begin{lstlisting}
# cp /boot/initramfs-$(uname -r).img /boot/initramfs-$(uname -r).img.bak
# dracut /boot/initramfs-$(uname -r).img $(uname -r)
# rm /boot/initramfs-$(uname -r).img.bak
\end{lstlisting}
Convert to text mode (runlevel 3), i.e., manipulate in command line window.
\begin{lstlisting}
# systemctl set-default multi-user.target
# init 3
# reboot	; this command is not necessary
\end{lstlisting}
Here, we find that the screen turns into command line with low resolution  apparently.~Switch to the directory where the driver file (NVIDIA-Linux-x86\_64-361.42.run) exists.
\begin{lstlisting}
# cd /to/your/directory/
# ./NVIDIA-Linux-x86_64-361.42.run --kernel-source-path=/usr/src/kernels/3.10.0-327.el7.x86_64  -k $(uname -r)
\end{lstlisting}
Steps as follows:
\begin{lstlisting}
LICENSE ... -accept
Building kernel modules ...
Install NVIDIA's 32-bit compatibility libraries. -yes
...
WARNING: Unable to perform the runtime configuration check for 32-bit library ... Assuming successful installation. -ok
Would you like to run the nvidia-xconfig utility to automatically update you X configuration file ... -yes -ok
\end{lstlisting}
Convert back to graphical mode (runlevel 5),
\begin{lstlisting}
# systemctl set-default graphical.target
# init 5
\end{lstlisting}
At this time, we will see the display shows graphical window with very high definition resolution.~Now check graphic driver has been successfully installed in our system.
\begin{lstlisting}
# lspci |grep -i vga
01:00.0 VGA compatible controller: NVIDIA Corporation GK208 [GeForce GT 640 Rev. 2] (rev al)
\end{lstlisting}

\section{Download RUN version of CUDA 7.5 toolkit}
Download it from its official website.~Choose Linux - x86\_64 - CentOS - 7 - runfile(local) -  Download(1.1 GB).

\section{Install full version of CUDA 7.5 toolkit (Toolkit, Samples)}
Fisrt, we choose `no' for Graphics Driver.~Caution ! Do not choose `yes' to install OpenGL, for which may conflicts with GNOME, causing  the crash of desktop.
\begin{lstlisting}
# ./cuda_7.5.18_linux.run
...
Install NVIDIA Accelerated Graphics Driver ... -n
Install OpenGL ... -n
Install CUDA 7.5 Toolkit ... -y
Toolkit location /usr/local/cuda-7.5 ... [Enter]
Install a symbolic link at ... -y
Install CUDA 7.5 Samples ... -y
Enter CUDA Samples Location ... [Enter]
...
Finished 
Driver : Not Selected
Toolkit : Installed in /usr/local/cuda-7.5
Samples : Installed in /root, but missing recommended libraries
***WARNING: Incomplete installation!
\end{lstlisting}
Remember the location of directory of NVIDIA\_CUDA-7.5\_Samples, for we need to check whether our CUDA  works below.

\section{Add path to environment variables}
Edit /etc/profile to update PATH and LD\_LIBRARY\_PATH.~Modify /etc/profile like this,
\begin{lstlisting}
...
export PATH=/usr/local/cuda-7.5/bin:$PATH
export LD_LIBRARY_PATH=/usr/local/cuda-7.5/lib64:$LD_LIBRARY_PATH
# source /etc/profile
\end{lstlisting}

\section{Upgrade build compilers}
Update G++/GCC compiler.
\begin{lstlisting}
# yum install build-essential
# gcc -v
# make -v
\end{lstlisting}

\section{Compile CUDA samples and check CUDA driver}
\begin{lstlisting}
# cd /to/your/path/NVIDIA_CUDA-7.5_Samples/
# make
\end{lstlisting}
After a long time compilation, none of errors occur, which suggests CUDA has been successfully installed.
\begin{lstlisting}
# nvcc -V
nvcc : NVIDIA (R) cuda compiler driver

# nvidia-smi
A table containing NVIDIA-SMI: 361.42 Driver Version: 361.42 appears.
GPU Name ...
Processes ...
\end{lstlisting}

\section{Install recommended packages and libraries}
Note that pip, nose and OpenBLAS are installed separately through .tar.gz files.~They may be downloaded from website such as https://github.com.
\begin{lstlisting}
# yum install numpy, scipy, python-devel, git
# tar -zxf nose-1.3.7.tar.gz
# cd nose-1.3.7/
# python setup.py install

# tar -zxf pip-8.1.2.tar.gz
# cd pip-8.1.2/
# python setup.py install

# tar -zxf  OpenBLAS-0.2.18.tar.gz
# cd OpenBLAS-0.2.18/
# make
\end{lstlisting}

\section{Install Theano}
Only a small library the Theano is, but it requires the network with an appropriate status to download it.
\begin{lstlisting}
# pip install Theano
\end{lstlisting}

\section{Make /root/.theanorc file}
\begin{lstlisting}
# vim /root/.theanorc
...
[global]
device=gpu
floatX=float32
root=/usr/local/cuda-7.5
[nvcc]
fastmath = True
[blas]
ldflags = -lopenblas
\end{lstlisting}
Save this file in path /root, and do not move it to other places.

\section{Test Theano library}
Run testing\_theano.py file (accompanied with this tutorial) using python.
\begin{lstlisting}
# sudo python testing_theano.py
...
Using gpu device 0: GeForce GT 640 ...
Looping 1000 times took 0.351630 seconds
Result is CudaNdarray ...
Numpy result is [...
Used the gpu
\end{lstlisting}
From the information printed above, we see python is able to use GPU to run the program now.

\section*{All complete ! So far We have successfully deployed CUDA 7.5 toolkit in our CentOS 7.}
\end{document}

